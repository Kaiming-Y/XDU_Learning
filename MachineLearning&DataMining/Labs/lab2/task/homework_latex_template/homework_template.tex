%%% 编译:XeLaTex. 
\documentclass[a4paper,UTF8]{article}
\usepackage{ctex}
\usepackage[margin=1.25in]{geometry}
\usepackage{color}
\usepackage{graphicx}
\usepackage{amssymb}
\usepackage{amsmath}
\usepackage{amsthm}
\usepackage{enumerate}
\usepackage{bm}
\usepackage{hyperref}
\numberwithin{equation}{section}
%\usepackage[thmmarks, amsmath, thref]{ntheorem}
\theoremstyle{definition}
\newtheorem*{solution}{解答}
\newtheorem*{prove}{Proof}
\newcommand{\indep}{\rotatebox[origin=c]{90}{$\models$}}

\usepackage{multirow}

%--

%--
\begin{document}
\title{机器学习与数据挖掘- 第三次作业\\ 选取最优划分准则构造决策树}
\author{学号:xxxxxx, 姓名:xxxx, 邮箱:xxxxxx@xxxxx.com}
\maketitle

\section{决策树-信息增益准则}
\begin{enumerate}[\textbf{题目 1:}] % {(}1{)}
\item 考虑下面的训练集:共计6个训练样本,每个训练样本有三个维度的特征属性和标记信息。详细信息如表1所示。

请通过训练集中的数据训练一棵决策树,要求通过“信息增益”(information gain)为准则来选择划分属性。请参考《机器学习》(周志华)书中图4.4,给出详细的计算过程并画出最终的决策树。

\begin{table}[htb]   
	\begin{center}   
		\caption{训练集信息}  
		\label{table:1} 
		\begin{tabular}{|c|c|c|c|c|} \hline   
\text{序号}  &  \text{特征 A}  & \text{特征 B}   & \text{特征 C}  &  \text{标记}  \\ \hline  
1  & 0 & 1 & 1  &0  \\ \hline
2  & 1 & 1 & 1  &0  \\ \hline
3  & 0 & 0 & 0  &0  \\ \hline
4  & 1 & 1 & 0  &1  \\ \hline
5  & 0 & 1 & 0  &1  \\ \hline
6  & 1 & 0 & 1  &1  \\ \hline
		\end{tabular}   
	\end{center}   
\end{table}
\end{enumerate}

%%%%%%%%%%%%  以下是解答  %%%%%%%%%%%%%%%%%%%%%%%%%%%%
\begin{solution}
此处用于写解答(中英文均可)\\
(1) xxxxxxxxx。\\
(2) xxxxxxxxx。\\
\end{solution}



\begin{enumerate}[\textbf{题目 2:}] % {(}1{)}
\item 模仿给出算法ID3的程序,实现算法C4.5在西瓜数据集2.0上训练的程序,给出训练完成后得到的决策树,并且给出在两个测试样本$test\_data\_1$ 和$test\_data\_2$ 上的分类预测结果。\\
$test\_data\_1$ = $\{$'色泽': '青绿', '根蒂': '蜷缩', '敲声': '浊响', '纹理': '稍糊', '脐部': '凹陷', '触感': '硬滑'$\}$\\
$test\_data\_2 $= $\{$'色泽': '乌黑', '根蒂': '稍蜷', '敲声': '浊响', '纹理': '清晰', '脐部': '凹陷', '触感': '硬滑'$\}$
\end{enumerate}


%%%%%%%%%%%%  以下是解答  %%%%%%%%%%%%%%%%%%%%%%%%%%%%
\begin{solution}
此处用于写解答(中英文均可)\\
(1) xxxxxxxxx。\\
(2) xxxxxxxxx。\\
\end{solution}


%\begin{prove}
%此处用于写证明(中英文均可)
%\end{prove}


\end{document}